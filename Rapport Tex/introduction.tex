\renewcommand{\abstractnamefont}{\normalfont\LARGE\bfseries}
%\renewcommand{\abstracttextfont}{\normalfont\Huge}


\chapter*{Introduction}
\begin{spacing}{1}
\addcontentsline{toc}{chapter}{Introduction}
\vspace{1cm}
Le commerce électronique (ou encore {\textit{e-commerce}}) est l'avenir incontestable de toute vente au détail, et l'industrie pharmaceutique ne déroge pas à la règle. Ce besoin est devenu plus flagrant pendant la pandémie de COVID-19, avec des milliers si ce n'est des millions de clients utilisant des plateformes de pharmacie en ligne au lieu de se rendre dans les commerces locaux. Le chiffre d'affaires du secteur des pharmacies en ligne devrait atteindre 22,59 milliards de dollars en 2024, ce qui signifie que l'industrie pharmaceutique ne cessera de se développer. En effet, le lancement d'une pharmacie en ligne est crucial car un site de commerce électronique indique la fiabilité de l'entreprise et sert d'outil efficace pour suivre le rythme de la concurrence.
\\ \\
C'est dans ce sillage que s'inscrit le projet marquant le dénouement du cours de Programmation Web et J2EE dispensé lors de nos études d'ingénierie des données, et lequel consiste en l'élaboration et l'implantation d'une plate-forme pharmaceutique électronique, moyennant les technologies J2EE.
\\ \\
Cette plate-forme propose à sa clientèle une large gamme de produits pharmaceutiques. Elle doit pour cela être en mesure de répondre autant que possible aux besoins de ses consommateurs.Par conséquent, le site doit mettre en évidence plusieurs rubriques de classement de médicaments avec leurs propriétés et leur utilisation ainsi qu’un onglet de recherche. Cette dernière option permet aux clients de retrouver facilement les produits dont ils ont besoin sans devoir défiler plusieurs pages.
\\ \\
Le présent rapport rend compte de tout ce qui a été accompli dans le cours de ce projet, tout en présentant les différents outils et modèles conceptuels utilisés. 

\end{spacing} 

\vspace*{\stretch{1}}

\newpage
\chapter*{Abstract}
\begin{spacing}{1}
\addcontentsline{toc}{chapter}{Abstract}
\vspace{1cm}
E-commerce is the future of any retail business, and the pharmaceutical industry is no exception. This became even more evident during the COVID pandemic, when more customers used online pharmacies instead of visiting local drug stores and waiting in line.
\\ \\
Our WEB/J2EE project is envisioned to have an impact on e-drugstores, consisting on developing a Web platform that allows shoppers to order medications at any time even while brick-and-mortar drugstores are closed. It contains principally two types of services :
\vspace{.3cm}
\begin{itemize}
\item General Recommendation for guest users.
\item A Favorites feature is crucial to nearly any pharmacy site, so users can quickly reorder their medications.
\item Information updates and promotions
\end{itemize}
\vspace{.3cm}
During a visit, users will have the option to choose Guest mode that offers a recommendation based on a description paragraph given by the user, or choose Authentication mode that can either be done through signing up to the platform or with a Google account.
\\ \\
More and more, customers expect the kinds of features that e-commerce provides. Leveraging these new technologies in our pharmacy business may help use stand out from competitors. An e-commerce solution for pharmacies will also reduce backorders by searching multiple warehouses for medication, and offer other benefits as well.


\end{spacing}

